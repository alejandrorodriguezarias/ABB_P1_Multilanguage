% !TEX TS-program = pdflatex
% !TEX encoding = UTF-8 Unicode

% This is a simple template for a LaTeX document using the "article" class.
% See "book", "report", "letter" for other types of document.

\documentclass[11pt]{scrartcl} % use larger type; default would be 10pt

\usepackage[utf8]{inputenc} % set input encoding (not needed with XeLaTeX)

%%% Examples of Article customizations
% These packages are optional, depending whether you want the features they provide.
% See the LaTeX Companion or other references for full information.

%%% PAGE DIMENSIONS
\usepackage{geometry} % to change the page dimensions
\geometry{a4paper} % or letterpaper (US) or a5paper or....
% \geometry{margin=2in} % for example, change the margins to 2 inches all round
% \geometry{landscape} % set up the page for landscape
%   read geometry.pdf for detailed page layout information

\usepackage{graphicx} % support the \includegraphics command and options

% \usepackage[parfill]{parskip} % Activate to begin paragraphs with an empty line rather than an indent

%%% PACKAGES
\usepackage{booktabs} % for much better looking tables
\usepackage{array} % for better arrays (eg matrices) in maths
\usepackage{paralist} % very flexible & customisable lists (eg. enumerate/itemize, etc.)
\usepackage{verbatim} % adds environment for commenting out blocks of text & for better verbatim
\usepackage{subfig} % make it possible to include more than one captioned figure/table in a single float
% These packages are all incorporated in the memoir class to one degree or another...

%%% HEADERS & FOOTERS
%\usepackage{fancyhdr} % This should be set AFTER setting up the page geometry
%\pagestyle{fancy} % options: empty , plain , fancy
%\renewcommand{\headrulewidth}{0pt} % customise the layout...
%\lhead{}\chead{}\rhead{}
%\lfoot{}\cfoot{\thepage}\rfoot{}

%%% SECTION TITLE APPEARANCE
\usepackage{sectsty}
\allsectionsfont{\sffamily\mdseries\upshape} % (See the fntguide.pdf for font help)
% (This matches ConTeXt defaults)

%%% ToC (table of contents) APPEARANCE
\usepackage[nottoc,notlof,notlot]{tocbibind} % Put the bibliography in the ToC
\usepackage[titles,subfigure]{tocloft} % Alter the style of the Table of Contents
\renewcommand{\cftsecfont}{\rmfamily\mdseries\upshape}
\renewcommand{\cftsecpagefont}{\rmfamily\mdseries\upshape} % No bold!

%%% END Article customizations

%%%% VERY CUSTOM CUSTOM CUSTOMIZATIONS %%%
\usepackage{sectsty}
\sectionfont{\rmfamily}
\subsectionfont{\rmfamily}

\usepackage{datetime}

\usepackage{enumitem}

\title{Deseño de Linguaxes de Programación}
\subtitle{Work report}
\author{Rodríguez Arias, Alejandro\\
	\texttt{@udc.es}
	\and Bouzas Quiroga, Jacobo\\
	\texttt{jacobo.bouzas.quiroga@udc.es}}

\makeatletter
\@ifpackagelater{scrbase}{2014/12/12}{}{\def\scr@startsection{\@startsection}}
\makeatother

% fecha
\newdate{release}{04}{10}{2017}
\date{\displaydate{release}}

%%% The "real" document content comes below...
\begin{document}
\maketitle
\clearpage
\tableofcontents
\clearpage

\section{Introduction}

This document is meant to display the work made on implementing the same data structure, a binary search tree, and associated algorithms in an array of assorted programming languages.

\section{Language: C}

\begin{description}[align=left,labelwidth=10em]
\item [Compiler] GCC 5.4.0
\item [Operating System] Ubuntu 16.04.2 LTS 64 bits
\end{description}

C is an imperative procedural programming language that supports structured programming and recursion. C has a weak and static typing and provides a low-level access to memory, allowing us to make a dinamic and manual memory management. C gives us a lot of control flow tools for our problem:

\begin{itemize}  
\item Executing a set of statements only if some condition is met (\texttt{if-else}).
\item Executing a set of statements zero or more times, until some condition is met (\texttt{while(condition)}). 
\item Allows a variable to be tested for equality against a list of values. Each value is called a case, and the variable being switched on is checked for each (\texttt{switch-case}).
\item Executing a set of distant statements, after which the flow of control usually returns (subroutines).
\end{itemize}

The C library \texttt{stdlib} provides us  functions to make a manual memory management of the dinamic memory. In our problem we use two of that functions:
\begin{itemize}
\item The \texttt{malloc(size)} function allocates “size” bytes and returns a pointer to the allocated memory.
\item The \texttt{free} function frees the memory space pointed by the pointer.
\end{itemize}
	
For the data structure we use \texttt{struct}. A user defined data type available in C that allows to combine data items of different kinds. Structures are used to represent a record. 

\subsection*{Memory Management}

The main differences when it comes to adapt Pascal code to C code are Pascal abstractions in memory management and reference parameters, since C has a lower level memory management.

	To inicialize a pointer in Pascal is enough to call the "new" function with the pointer as a parameter, but in order to initialize a pointer in C
	setting the pointer to the returned value of the "malloc" function is needed. The "malloc" function requires the size of the pointer in bytes to allocate
	the memory.

Differences within reference parameters

	Pascal has a "VAR" keyword to pass arguments by reference but in C language every argument is passed by value, so passing a pointer to the memory address
	of the argument is needed in order to modify it.



Unit in C


	C needs two files to make a library: a code file (.c) and a header file (.h). The header file provides a program with library functions.
	Pascal has the header and source code in the same file.

\section{Language: Java}
\section{Language: Ruby}
\section{Language: OCaml}

\end{document}
